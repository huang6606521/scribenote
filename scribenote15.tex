
\documentclass[11pt]{article}
\usepackage{latexsym}
\usepackage{amsmath}
\usepackage{amssymb}
\usepackage{amsthm}
\usepackage{epsfig}
\usepackage{graphicx}
\usepackage{algorithm}
\usepackage{algorithmic}

\newcommand{\handout}[5]{
  \noindent
  \begin{center}
  \framebox{
    \vbox{
      \hbox to 5.78in { {\bf ICS 443: Parallel Algorithms} \hfill #2 }
      \vspace{4mm}
      \hbox to 5.78in { {\Large \hfill #5  \hfill} }
      \vspace{2mm}
      \hbox to 5.78in { {\em #3 \hfill #4} }
    }
  }
  \end{center}
  \vspace*{4mm}
}

\newcommand{\lecture}[4]{\handout{#1}{#2}{#3}{Scribe: #4}{Lecture #1}}

\newtheorem{theorem}{Theorem}
\newtheorem{corollary}[theorem]{Corollary}
\newtheorem{lemma}[theorem]{Lemma}
\newtheorem{observation}[theorem]{Observation}
\newtheorem{proposition}[theorem]{Proposition}
\newtheorem{definition}[theorem]{Definition}
\newtheorem{claim}[theorem]{Claim}
\newtheorem{fact}[theorem]{Fact}
\newtheorem{assumption}[theorem]{Assumption}

% 1-inch margins, from fullpage.sty by H.Partl, Version 2, Dec. 15, 1988.
\topmargin 0pt
\advance \topmargin by -\headheight
\advance \topmargin by -\headsep
\textheight 8.9in
\oddsidemargin 0pt
\evensidemargin \oddsidemargin
\marginparwidth 0.5in
\textwidth 6.5in

\parindent 0in
\parskip 1.5ex
%\renewcommand{\baselinestretch}{1.25}

\begin{document}

\lecture{15 --- October 11, 2017}{Fall 2017}{Prof.\ Nodari Sitchinava}{Jipeng Huang, Christopher Yeager}

\section{Homework 2 Solutions}

\subsection{Problem 1}

\begin{algorithm}
\begin{algorithmic}
\FOR{$i = 1~to~p {\bf~in~parallel}$}
    \FOR{$j = (i-1)\frac{n}{p}+2~to~i\frac{n}{p}$}
        \STATE $A[i] = A[i-1] + A[i]$
    \ENDFOR
\ENDFOR
\FOR{$i = 1~to~p {\bf~in~parallel}$}
    \STATE $B[i] = A[i * \frac{n}{p}]$
    \STATE $prefix-sum(B)$
\ENDFOR
\FOR{$i = 2~to~p {\bf~in~parallel}$}
    \FOR{$j = (i-1)\frac{n}{p}+1~to~i\frac{n}{p}$}
        \STATE $A[j] = B[i-1] + A[j]$
    \ENDFOR
\ENDFOR
\end{algorithmic}
\end{algorithm}

\subsection{Problem 2}
As we know, the total coins are $(C_1 +C_2 +...+C_n )$ and there are m robbers which each robber share equally. Therefore, each robber should have total coins divided by m robbers.  We allocate an array $A[n]$ to calculate the prefix-sum of coins so the $share = \lfloor \frac{A[n]}{m} \rfloo$ and $ith$ robber will take $(i-1) * share + 1$ to $i * share$ coins. 

\begin{algorithm}[H]
\caption{Split-loot(C, n , m)}
\begin{algorithmic}
\STATE $A[0...n]$
\STATE $C[1...n]$
\FOR{$ i = 1~to~n {\bf~in~parallel}$}
    \STATE $A[i] = C[i]$
\ENDFOR
\STATE $prefix-sum(A)$
\STATE $share = \lfloor \frac{A[n]}{m} \rfloo$
\STATE  $A[0] = 0$
\FOR{$ j = 1~to~m {\bf~in~parallel}$}
    \STATE $output[j].l = predecessor((j-1)*share+1,A)$
    \STATE $output[j].r = predecessor(j*share+1,A)$
\ENDFOR
\IF{$A[output[j].l != (j-1)*share+1]$}
    \STATE $output[j].l ++$
\ENDIF
\IF{$A[output[j].r != (j-1)*share]$}
    \STATE $output[j].r ++$
\ENDIF
\FOR{$ j = 1~to~m {\bf~in~parallel}$}
    \STATE $output[j].t_l = min{share, A[output[j].l]-(j-1)*share}$
    \IF{$output[j].t_l == share$}
        \STATE $output[j].r ++$
        \STATE $output[j].t_r = 0$
    \ELSE
        \STATE $output[j].t_r = j * share - A[output[j].r-1]$
    \ENDIF
\ENDFOR
\end{algorithmic}
\end{algorithm}
We analyzed this algorithm that first for loop takes constant time and $O(n)$ work. The $prefix-sum(A)$ we talked in previous class and it takes $O(\log n)$ time and $O(n)$ work.
Each $predecessor$ function takes $O(1)$ time and $O(\log n)$ work.  Therefore, the total time is $O(\log n)$ and work is $O(n)$ work.

\subsection{Problem 3}
We assume $a<b<c$ then we got:
$min (min (a,b),c) = min(a,c) = a$ and
$min (min a,(b,c)) = min(a,b) = a$.\\
Assume $a<c<b$ and we got:
$min (min (a,b),c) = min(a,c) = a$ and
$min (min a,(b,c)) = min(a,c) = a$.\\
Assume $b<a<c$ and we got:
$min (min (a,b),c) = min(b,c) = b$ and
$min (min a,(b,c)) = min(a,b) = b$.\\
Assume $b<c<a$ and we got:
$min (min (a,b),c) = min(b,c) = b$ and
$min (min a,(b,c)) = min(a,b) = b$.\\
Assume $c<a<b$ and we got:
$min (min (a,b),c) = min(a,c) = c$ and
$min (min a,(b,c)) = min(a,c) = c$.\\
Assume $c<b<a$ and we got:
$min (min (a,b),c) = min(b,c) = c$ and
$min (min a,(b,c)) = min(a,c) = c$.\\
Therefore, the min is associative operator.
since $min(I_{min},x) = x$, so its identity is $I_{min} = \infty$

\begin{algorithm}
\caption{Minima(A[0...n-1])}
\begin{algorithmic}
\IF{$n == 0$}
    \STATE $return A[0]$
\ELSE
    \FOR{$i =0~to~\frac{n}{2}-1 {\bf~in~parallel}$}
        \STATE $B[i] = min(A[2i], A[2i+1])$
    \ENDFOR
    \STATE $Minima(B[0...\frac{n}{2}-1])$
\ENDIF

\end{algorithmic}
\end{algorithm}

%\bibliography{mybib}
\bibliographystyle{alpha}

\end{document}
